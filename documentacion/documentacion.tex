% !TeX spellcheck = es_ES
\documentclass[a4paper,12pt]{article}
\usepackage[utf8]{inputenc}
\usepackage[spanish]{babel}
\usepackage{listings}
\usepackage{color}
\usepackage{hyperref}

\definecolor{codegreen}{rgb}{0,0.6,0}
\definecolor{codegray}{rgb}{0.5,0.5,0.5}
\definecolor{codepurple}{rgb}{0.58,0,0.82}
\definecolor{backcolour}{rgb}{0.95,0.95,0.92}
 
\lstdefinestyle{mystyle}{
    backgroundcolor=\color{backcolour},   
    commentstyle=\color{codegreen},
    morekeywords={let, function},
    keywordstyle=\color{magenta},
    numberstyle=\tiny\color{codegray},
    stringstyle=\color{codepurple},
    basicstyle=\footnotesize,
    breakatwhitespace=false,         
    breaklines=true,                 
    captionpos=b,                    
    keepspaces=true,                 
    numbers=left,                    
    numbersep=5pt,                  
    showspaces=false,                
    showstringspaces=false,
    showtabs=false,                  
    tabsize=2
}
 
\lstset{style=mystyle}

%opening
\title{}
\title{Documentación del proyecto}
\author{Barrera Pérez Carlos Tonatihu \\ Profesor: Eduardo Gutiérrez Aldana \\ Administración de servicios en red \\ Grupo: 4CM2 }

\begin{document}

\maketitle
\newpage
\tableofcontents
\newpage

\section{Descripción del proyecto}
El proyecto consiste de una topología de red en la cual se presentan enrutadores, switches, servidores, gestor y clientes. Los enrutadores y servidor son monitoreados por el gestor con el objetivo de llevar un registro de los datos que estos presentan y poder detectar fallas en los equipos.

\subsection{Herramientas utilizadas}
El proyecto fue desarrollado en el ambiente de simulación \emph{GNS3} que se encuentra integrado dentro del sistema operativo \emph{Live Raizo - Linux for Virtual SysAdmin} en su versión \emph{8.17.08.30p} (\url{https://sourceforge.net/projects/live-raizo/files/}) ya que es la ultima versión que incluye \emph{Virtual Box} que se utilizo para trabajar con las maquinas virtuales que se trabaron.

Para poder trabajar de forma adecuada se adecuo una USB a modo de arranque con \emph{Raizo} y para evitar la perdida del trabajo y guardar los datos que se encuentran en \textbf{/home} y \textbf{/opt} se creo una partición en el disco duro de la computadora de trabajo. Esta partición se formateo como \textbf{ext4} y se le etiqueto como \textbf{persistence} y se le agrego un archivo llamado \textbf{persistence.conf} con la siguiente información.

\begin{lstlisting}[language=bash]
 /home
 /opt union
 /root union
\end{lstlisting}

Es importante mencionar que este método de crear persistencia se puede realizar sobre una USB y tener una USB para el boot y otra para la persistencia o ambas características en una sola USB.

En el DVD que se entregó se encuentran las carpetas \textbf{home}, \textbf{opt} con las maquinas virtuales que se trabajaron y el archivo \textbf{persistence.conf}.

De manera general otras herramientas que se utilizaron fueron \href{http://rcp100.sourceforge.net/}{RCP100} y \href{https://distro.ibiblio.org/tinycorelinux/}{TinyCore} como maquinas virtuales para los enrutadores, el gestor y el servidor.

\section{Desarrollo}
\subsection{Servidor}
El servidor web fue implementado en \emph{Tiny Core} utilizando el paquete \emph{Busybox HTTPD}, el codigo que se encarga de esto se encuentra en el archivo \textbf{/opt/bootlocal.sh} para que cada vez que se encienda la computadora el servidor comience a trabajar. El código es el siguiente.

\begin{lstlisting}[language=bash]
 /usr/local/httpd/sbin/httpd -p 80 -h /home/tc/site
\end{lstlisting}
Lo que hace este código es activar el servicio de HTTP proporcionado por \emph{Busybox} en el puerto 80 indicando que la página a mostrar (index.html) se encuentra en la carpeta \textbf{/home/tc/site}.
Además, se debe de configurar una interfaz de la maquina virtual para que se pueda acceder al servidor.
\subsection{Enrutador}
Para el enrutador se utilizo RCP100, en cada uno de los enrutadores se configuraron las interfaces y el enrutamiento dinámico OSPF. Además de configurar el agente snmp necesario para llevar a cabo el monitoreo de interfaces, memoria y uso de CPU, esta configuración se realizo de la siguiente forma.
\begin{lstlisting}[language=bash]
 snmp-server contact TalesDeMileto
 snmp-server location Topolovampo
 snmp-server community soporte ro
\end{lstlisting}
\subsection{Gestor}
\subsubsection{Ping}
\begin{lstlisting}[language=bash]
 echo "Hola"
\end{lstlisting}


\end{document}
